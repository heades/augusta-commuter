M\begin{definition}
  \label{def:exp-action}
  A \emph{linear category}, $\cat{C}$, consists of the following structure:
  \begin{itemize}
  \item A is a symmetric monoidal category $(\mathcal{A},\otimes,I,\lambda,\rho,\alpha,\beta)$,
  \item A linear exponential comonad, $(!_{-},\delta,\varepsilon)$,
    which has the following structure:
    
    \begin{itemize}    
    \item The functor $! : \mathcal{A} \mto \mathcal{A}$ forms comonad
      on $\cat{A}$.  That is, there are two natural transformations
      $\delta : ! A \mto !! A$ and $\varepsilon : !A \mto A$ that make
      the following diagrams commute:
      \begin{center}
        \begin{math}
          \begin{array}{cccc}
            \bfig
            \square|amma|/->`->`->`->/<1000,500>[
              !A`
              !!A`
              !!A`
              !!!A;
              \delta`
              \delta`
              \delta`
              !\delta]
            \efig
            & \quad &
            \bfig
            \btriangle|mma|/->`=`->/<600,500>[
              !A`
              !!A`
              !A;
              \delta``
              \varepsilon]

            \qtriangle|amm|/->`=`->/<600,500>[
              !A`
              !!A`
              !A;
              \delta``
              !\varepsilon]
            \efig
          \end{array}
        \end{math}
      \end{center}    

    \item Four natural transformations:
      \begin{itemize}
      \item (Monoidal Map) $\mm_{A,B} : ! A \otimes ! B \mto ! (A \otimes B)$
      \item (Monoidal Unit Map) $\mm_I : I \mto ! I$
      \item (Contraction) $\con : !A \mto !A \otimes !A$
      \item (Weakening) $\weak : ! A \mto I$
      \end{itemize}

    \item The functor $! : \mathcal{A} \mto \mathcal{A}$ is symmetric
      monoidal.  That is, the following diagrams must commute:
      \begin{center}
        \begin{math}
          \begin{array}{c}
            \begin{array}{llllllll}
              \bfig
              \square|amma|/->`<-`->`->/<1000,500>[
                !I \otimes !A`
                !(I \otimes A)`
                I \otimes !A`
                !A;
                \mm`
                \nm \otimes \id`
                !\lambda`
                \lambda]
              \efig
              & \quad &
              \bfig
              \square|amma|/->`<-`->`->/<1000,500>[
                !A \otimes !I`
                !(A \otimes I)`
                !A \otimes I`
                !A;
                \mm`
                \id \otimes \nm`
                !\rho`
                \rho]
              \efig              
            \end{array}
            \\
            \bfig
            \square|amma|/->`->`->`->/<1000,500>[
              !A \otimes !B`
              !(A \otimes B)`
              !(B) \otimes !(A)`
              !(B \otimes A);
              \mm`
              !\beta`
              \beta`
              \mm]
            \efig                        
            \\
            \bfig
            \square|amma|/->`<-``->/<1500,500>[
              (!A \otimes !B) \otimes !C`
              !(A \otimes B) \otimes !C`
              !A \otimes (!B \otimes !C)`
              !A \otimes !(B \otimes C);
              \mm \otimes \id`
              \alpha``
              \id \otimes \mm]

            \square(1500,0)|amma|/->``<-`->/<1500,500>[
              !(A \otimes B) \otimes !C`
              !((A \otimes B) \otimes C)`
              !A \otimes !(B \otimes C)`
              !(A \otimes (B \otimes C));
              \mm``
              !\alpha`
              \mm]
            \efig            
          \end{array}
        \end{math}
      \end{center}

    \item The functor $! : \cat{A} \mto \cat{A}$ is a symmetric
      monoidal comonad.  That is, the following diagrams must commute:
      \begin{center}
        \begin{math}
          \begin{array}{ccc}
            \bfig
            \square|amma|/->`->``/<1000,500>[
              !A \otimes !B`
              !(A \otimes B)`
              !!A \otimes !!B`;
              \mm`
              \delta \otimes \delta``]

            \square(0,-500)|amma|/`->``->/<1000,500>[
              !!A \otimes !!B``
              !(!A \otimes !B)`
              !!(A \otimes B);`
              \mm``
              !\mm]

            \morphism(1000,500)|m|<0,-1000>[
              !(A \otimes B)`
              !!(A \otimes B);
              \delta]
            \efig
            & \quad &
            \bfig
            \qtriangle|amm|<1000,500>[
              !A \otimes !_1B`
              !(A \otimes B)`
              A \otimes B;
              \mm`
              \varepsilon \otimes \varepsilon`
              \varepsilon]
            \efig\\\\
            
            \bfig
            \square|amma|<1000,500>[
              I`
              !I`
              !I`
              !!I;
              \nm`
              \nm`
              \delta`
              !\nm]
            \efig
            & \quad &
            \bfig
            \qtriangle|amm|/->`=`->/<1000,500>[
              I`
              ! I`
              I;
              \nm``
              \varepsilon]
            \efig
          \end{array}
        \end{math}
      \end{center}

    \item Weakening must satisfy the following diagrams:
      \begin{center}
        \begin{math}
          \begin{array}{ccc}
            \bfig
            \square|amma|<1000,500>[
              !A \otimes !B`
              !(A \otimes B)`
              I \otimes I`
              I;
              \mm`
              \weak \otimes \weak`
              \weak`
              \lambda = \rho]
            \efig
            & \quad &
            \bfig
            \qtriangle|amm|/->`=`->/<1000,500>[
              I`
              !I`
              I;
              \nm``
              \weak]
            \efig
          \end{array}
        \end{math}
      \end{center}

    \item Contraction must satisfy the following diagrams:
      \begin{center}
        \begin{math}
          \begin{array}{ccc}
            \bfig
            \square|amma|/->`->``/<1500,500>[
              !A \otimes !B`
              !(A \otimes B)`
              (!A \otimes !A) \otimes (!B \otimes !B)`;
              \mm`
              \con \otimes \con``]

            \square(0,-500)|amma|/`->``->/<1500,500>[
              (!A \otimes !A) \otimes (!B \otimes !B)``
              (!A \otimes !B) \otimes (!A \otimes !B)`
              !(A \otimes B) \otimes !(A \otimes B);`
              \cong``
              \mm \otimes \mm]

            \morphism(1500,500)|m|<0,-1000>[
              !(A \otimes B)`
              !(A \otimes B) \otimes !(A \otimes B);
              \con]
            \efig
            & \quad &
            \bfig
            \square|amma|<1000,500>[
              I`
              !I`
              I \otimes I`
              !I \otimes !I;
              \nm`
              \lambda^{-1} = \rho^{-1}`
              \con`
              \nm \otimes \nm]
            \efig
          \end{array}
        \end{math}
      \end{center}

    \item Weakening and contraction form a commutative comonoid.  That
      is, the following diagrams commute:
      \begin{center}
        \begin{math}
          \begin{array}{c}
            \begin{array}{lllll}
              \bfig
              \btriangle|amm|/->`->`<-/<1000,500>[
                !A`
                I \otimes !A`
                !A \otimes !A;
                \lambda^{-1}`
                \con`
                \weak \otimes \id]

              \qtriangle|amm|/->``<-/<1000,500>[
                !A`
                !A \otimes I`
                !A \otimes !A;
                \rho^{-1}``
                \id \otimes \weak]              
              \efig              
              & \quad &
              \bfig

              \btriangle|mma|/->`->`->/<1000,500>[
                !A`
                !A \otimes !A`
                !A \otimes !A;
                \con`
                \con`
                \beta]              
              \efig
            \end{array}\\
            \bfig
            \square|amma|/->`=``->/<1500,500>[
              !A`
              !A \otimes !A`
              !A`
              !A \otimes !A;
              \con```
              \con]

            \square(1500,0)|amma|/->``<-`->/<1500,500>[
              !A \otimes !A`
              !A \otimes (!A \otimes !A)`
              !A \otimes !A`
              (!A \otimes !A) \otimes !A;
              \id \otimes \con``
              \alpha^{-1}`
              \con \otimes \id]
            \efig
          \end{array}
        \end{math}
      \end{center}            

    \item Weakening and contraction are coalgebra morphisms.  That is,
      the following diagrams must commute:
      \begin{center}
        \begin{math}
          \begin{array}{lll}
            \bfig
            \square|amma|/=`->``/<1000,500>[
              !A`
              !A`
              !!A`;`
              \delta``]

            \square(0,-500)|amma|/`->``<-/<1000,500>[
              !!A``
              !I`
              I;`
              !\weak``
              \nm]

            \morphism(1000,500)|m|<0,-1000>[
              !A`
              I;
              \weak]
            \efig
            & \quad &
            \bfig
            \square|amma|/=`->`->`/<1000,500>[
              ! A`
              ! A`
              ! ! A`
              !A \otimes !A;`
              \delta`
              \con`]

            \square(0,-500)|amma|/`->`->`<-/<1000,500>[
              ! ! A`
              !A \otimes ! A`
              !(!A \otimes !A)`
              !!A \otimes !!A;`
              ! \con`
              \delta \otimes \delta`
              \mm]
            \efig
          \end{array}
        \end{math}
      \end{center}
      
    \item $\delta_A : !A \mto !!A$ is a comonoid morphism between the
      comonoids $(!A,\weak,\con)$ and $(!!A,\weak,\con)$.  That is,
      the following diagrams must commute:
      \begin{center}
        \begin{math}
          \begin{array}{lll}            
            \bfig
            \btriangle|mma|/->`->`->/<1000,500>[
              !A`
              !!A`
              I;
              \delta`
              \weak`
              \weak]            
            \efig
            & \quad &
            \bfig
            \square|amma|/=`->``/<1000,500>[
              !A`
              !A`
              !!A`;`
              \delta``]
            
            \square(0,-500)|amma|/`->``<-/<1000,500>[
              !!A``
              !!A \otimes !!A`
              !A \otimes !A;`
              \con``
              \delta \otimes \delta]
            
            \morphism(1000,500)|m|<0,-1000>[
              !A`
              !A \otimes !A;
              \con]
            \efig
          \end{array}
        \end{math}
      \end{center}                
    \end{itemize}
  \end{itemize}    
\end{definition}

\begin{lemma}
  \label{lemma:two-b}
  Whenever $f : (!A,\delta_A) \mto (!B,\delta_B)$ is a coalgebra morphism between free coalgebras, then it is also a comonoid morphism.
\end{lemma}
\begin{proof}
  Suppose $f : (!A,\delta_A) \mto (!B,\delta_B)$ is a coalgebra
  morphism between free coalgebras.  This assumption amounts to
  requiring that the following diagram commutes:
  \begin{center}
    \begin{math}
      \bfig
      \square|amma|/->`->`->`->/<500,500>[
        !A`
        !B`
        !!A`
        !!B;
        f`
        \delta`
        \delta`
        !f]
      \efig
    \end{math}
  \end{center}
  It suffices to show that $f : (!A,\delta_A) \mto (!B,\delta_B)$ is
  also a comonoid morphism. Hence, we must show that the following
  diagrams commute:
  \begin{center}
    \begin{math}
      \begin{array}{lll}
        \bfig
        \square|amma|/->`->`=`->/<500,500>[
          !A`
          I`
          !B`
          I;
          \weak`
          f``
          \weak]
        \efig
        & \quad &
       \bfig
       \square|amma|/->`->`->`->/<700,500>[
         !A`
         !A \otimes !A`
         !B`
         !B \otimes !B;
         \con`
         f`
         f \otimes f`
         \con]
       \efig
      \end{array}
    \end{math}
  \end{center}
  The left diagram commutes, because the following expanded one does:
  \begin{center}
    \begin{math}
      \bfig
      \square|amma|/->`->`=`->/<2000,1300>[
        !A`
        I`
        !B`
        I;
        \weak`
        f``
        \weak]

      \Atriangle|mma|/<-`->`->/<1000,500>[
        !!B`
        !B`
        I;
        \delta`
        \weak`
        \weak]
      
      \morphism(1000,500)|m|/<-/<0,500>[
        !!B`
        !!A;
        !f]

      \Vtriangle(0,1000)|amm|/->`->`<-/<1000,300>[
        !A`
        I`
        !!A;`
        \delta`
        \weak]

      \place(500,700)[(2)]
      \place(1500,700)[(3)]
      \place(1000,250)[(1)]
      \place(1000,1170)[(4)]
      \efig
    \end{math}
  \end{center}
  Diagrams $(1)$ and $(4)$ commute because $\delta$ is a comonoid
  morphism, diagram $(2)$ commutes because $f$ is assumed to be a
  coalgebra morphism, and diagram $(3)$ commutes by naturality of
  $\weak$.  Note that we have numbered the previous diagram in the
  order of the necessary replacements needed when doing the same proof
  equationally, and we do the same for the next diagram.  This should
  make it easier to reconstruct the proof.
  
  \ \\ \noindent
  The diagram for contraction commutes (the diagram on the right
  above), because the following expanded one does:
  \begin{center}
    \begin{math}
      \bfig
      \square|amma|/->`->`->`->/<3000,2000>[
        !A`
        !A \otimes !A`
        !B`
        !B \otimes !B;
        \con`
        f`
        f \otimes f`
        \con]

      \square(300,350)|amma|/->`->`->`->/<2300,1300>[
        !!A`
        !(!A \otimes !A)`
        !!B`
        !(!B \otimes !B);
        !\con`
        !f`
        !(f \otimes f)`
        !\con]

      \square(800,650)|amma|/->`->`->`->/<1200,700>[
        !!A`
        !!A \otimes !!A`
        !!B`
        !!B \otimes !!B;
        \con`
        !f`
        !f \otimes !f`
        \con]

      \morphism(0,2000)|m|/->/<300,-350>[
        !A`
        !!A;
        \delta]

      \morphism(0,0)|m|/->/<300,350>[
        !B`
        !!B;
        \delta]

      \morphism(300,350)|m|/=/<500,300>[
        !!B`
        !!B;]

      \morphism(300,1650)|m|/=/<500,-300>[
        !!A`
        !!A;]

      \morphism(2000,1350)|m|/->/<600,300>[
        !!A \otimes !!A`
        !(!A \otimes !A);
        \mm]

      \morphism(2000,650)|m|/->/<600,-300>[
        !!B \otimes !!B`
        !(!B \otimes !B);
        \mm]

      \morphism(2600,350)|m|/->/<400,-350>[
        !(!B \otimes !B)`
        !B \otimes !B;
        \varepsilon]

      \morphism(2600,1650)|m|/->/<400,350>[
        !(!A \otimes !A)`
        !A \otimes !A;
        \varepsilon]

      \place(1400,200)[(1)]
      \place(1400,550)[(2)]
      \place(150,1000)[(3)]
      \place(1400,1000)[(4)]
      \place(2300,1000)[(5)]
      \place(2820,1000)[(6)]
      \place(1400,1540)[(7)]
      \place(1400,1850)[(8)]
      \efig
    \end{math}
  \end{center}
  Diagram $(3)$ commutes because $f$ is assumed to be a coalgebra
  morphism, diagram $(4)$ commutes by naturality of $\con$, diagram
  $(5)$ commutes by naturality of $\mm$, and diagram $(6)$ commutes by
  naturality of $\varepsilon$.

  \ \\ \noindent
  Diagrams $(1)$ and $(8)$ commute, because the following one does:
  \begin{center}
    \begin{math}
      \bfig
      \square|amma|/->`->`<-`->/<2000,700>[
        !X`
        !X \otimes !X`
        !!X`
        !(!X \otimes !X);
        \con`
        \delta`
        \varepsilon`
        !\con]

      \Vtriangle(0,350)|aam|/->`=`<-/<1000,350>[
        !X`
        !X \otimes !X`
        !X;``
        \con]

      \morphism|m|/->/<1000,350>[
        !!X`
        !X;
        \varepsilon]
      \efig
    \end{math}
  \end{center}
  The left triangle commutes, because $! : \cat{A} \mto \cat{A}$ is a
  comonad, and the bottom diagram commutes by naturality of
  $\varepsilon$.

  \ \\ \noindent
  Finally, diagrams $(2)$ and $(7)$ do not actually commute, but are
  parallel morphisms whose cofork is $\delta$.  That is, the following
  diagram commutes:
  \begin{center}
    \begin{math}
      \bfig
      \square|amma|/->`->`->`/<2000,700>[
        !X`
        !!X`
        !!X`
        !(!X \otimes !X);
        \delta`
        \delta`
        !\con`]

      \Atrianglepair|amaaa|/`->``->`->/<1000,400>[
        !X \otimes !X`
        !!X`
        !!X \otimes !!X`
        !(!X \otimes !X);`
        \delta \otimes \delta``
        \con`
        \mm]

      \morphism(0,700)|m|/->/<1000,-300>[
        !X`
        !X \otimes !X;
        \con]
      \efig
    \end{math}
  \end{center}
  The left diagram commutes because $\delta : !X \mto !!X$ is a
  comonoid morphism, and the right diagram commutes because weakening
  and contraction are coalgebra morphisms.

  \ \\\noindent
  Therefore, the original diagram above corresponds to the following
  equational proof:
  \begin{center}
    \begin{math}
      \begin{array}{llllll}
        f;c
        & = & f;\delta;!\con;\varepsilon                  & \text{Diagram }(1)\\
        & = & f;\delta;\con;\mm;\varepsilon               & \text{Diagram }(2)\\
        & = & \delta;!f;\con;\mm;\varepsilon              & \text{Diagram }(3)\\
        & = & \delta;\con;(!f \otimes !f);\mm;\varepsilon & \text{Diagram }(4)\\
        & = & \delta;\con;\mm;!(f \otimes f);\varepsilon  & \text{Diagram }(5)\\
        & = & \delta;\con;\mm;\varepsilon;(f \otimes f)   & \text{Diagram }(6)\\
        & = & \delta;!\con;\varepsilon;(f \otimes f)      & \text{Diagram }(7)\\
        & = & \con;(f \otimes f)                          & \text{Diagram }(8)\\
      \end{array}
    \end{math}
  \end{center}
\end{proof}
